\documentclass[aspectratio=169]{beamer}
\usepackage{graphicx}
\usepackage{listings}
\usepackage{hyperref}

\usetheme{metropolis}
\title{git/GitHub For Developers}
\institute{Engineers for Exploration, UC San Diego}
\setbeamertemplate{caption}[numbered]
\lstset{
    basicstyle=\ttfamily
}
\logo{\includegraphics[height=.65cm,keepaspectratio]{e4e_logo_350x136.png}}
\begin{document}
\maketitle

\begin{frame}{Introduction: Why do we care}
    \begin{center}
        \includegraphics[]{why_git.png}
    \end{center}
\end{frame}

\begin{frame}{Introduction: Track Changes, Not Files}
    \begin{center}
        \includegraphics[]{git_graph.png}
    \end{center}
\end{frame}

\begin{frame}{Introduction: Git GitHub}
    \begin{itemize}
        \item \textbf{Git vs. GitHub:} Distributed VCS vs. collaboration platform.
        \item \textbf{Purpose:} Enhances project management and teamwork.
        \item \textbf{Git Helps By:} tracking and saving verision changes over time
    \end{itemize}
    \begin{center}
        \includegraphics[scale=.25]{git_commands.png}
    \end{center}
\end{frame}

\begin{frame}{Tags}
    \begin{itemize}
        \item Marks significant project milestones.
        \item Useful for release points, use semantic versioning.
        \item Lightweight tags: \textit{git tag tagname}.
        \item Annotated tags: \textit{git tag -a tagname ``message''}.
        \item List/delete tags: \textit{git tag}, \textit{git tag -d tagname}.
    \end{itemize}
\end{frame}
\begin{frame}{Git Workflows}
    \begin{itemize}
        \item Feature Branch Workflow
        \item Gitflow Workflow
        \item Fork Workflow
    \end{itemize}
\end{frame}
\begin{frame}{Feature Branch Workflow}
    \begin{itemize}
        \item Develop each feature in its own branch.
        \item Merges via pull requests for code review.
        \item Keeps main branch stable, encourages collaboration.
        \item Ideal for projects with simultaneous feature development.
    \end{itemize}
    \begin{center}
        \includegraphics[scale=.25]{feature_workflow_diagram.png}
    \end{center}
\end{frame}
\begin{frame}{Gitflow Workflow}
    \begin{itemize}
        \item Structured model: development, features, releases, hotfixes.
        \item Systematic release management, clear branch roles.
        \item Tracks progress efficiently, supports parallel releases.
        \item Suited for scheduled release cycles.
    \end{itemize}
    \begin{center}
        \includegraphics[scale=.2]{gitflow_workflow_diagram.png}
    \end{center}
\end{frame}
\begin{frame}{Fork Workflow}
    \begin{itemize}
        \item Developers work on personal repository copies.
        \item Changes proposed via pull requests.
        \item Encourages external contributions, safe experimentation.
        \item Ideal for open-source and large collaborations.
    \end{itemize}
    \begin{center}
        \includegraphics[scale=.25]{fork_workflow_diagram.png}
    \end{center}
\end{frame}

\begin{frame}{.gitignore}
    \begin{center}
        \includegraphics[scale=.35]{gitignore_talk.png}
    \end{center}
\end{frame}


\begin{frame}{UI or CLI?}
    \begin{itemize}
        \item UI's exist to wrap around git
        \item CLI $\rightarrow$ Helps better understand whats happening
        \item CLI $\rightarrow$ May only have access to CLI
        \item UI $\rightarrow$ Useful for manging many repos
        \item UI $\rightarrow$ Visualization is nice 
    \end{itemize}
\end{frame}

\begin{frame}{Whatabout GitHub?}
    \begin{itemize}
        \item \textbf{Git vs. GitHub:} Distributed VCS vs. collaboration platform.
        \item \textbf{Purpose:} Enhances project management and teamwork.
        \item \textbf{GitHub helps by:} Being a collaborator platform built upon being a remote repo
    \end{itemize}
    \begin{center}
        \includegraphics[scale=.25]{git_commands.png}
    \end{center}
\end{frame}

\begin{frame}{Whatabout GitHub?}
    \begin{itemize}
        \item \href{https://github.com/UCSD-E4E/website2.0}{https://github.com/UCSD-E4E/website2.0}
    \end{itemize}
    \begin{center}
        \includegraphics[scale=.35]{github.png}
    \end{center}
\end{frame}

\begin{frame}{Report Issues}
    \begin{itemize}
        \item Tracks Bugs, Features Ideas, Discussions
        \item Assign people and PRs to address
        \item \href{https://github.com/UCSD-E4E/website2.0/issues/142}{https://github.com/UCSD-E4E/website2.0/issues/142}
    \end{itemize}
\end{frame}

\begin{frame}{Pull Requests (PR) Management}
    \begin{itemize}
        \item Handles merging branches from collaborators together
        \item Keep PRs small for easy review.
        \item Automate tests and checks via GitHub Actions.
        \item \href{https://github.com/UCSD-E4E/website2.0/pull/135}{https://github.com/UCSD-E4E/website2.0/pull/135}
        \item Most of the time merges work... but sometimes
    \end{itemize}
\end{frame}

\begin{frame}{Merging a Merge Conflict: Do's and Don'ts}
    \begin{itemize}
        \item Conflicts occur when a branch does not have some newer changes
        \item \textbf{don't try fixing a merge conflict without talking to your lead}
        \item anytime you are about to change git history, talk to a lead
        \item Here is how to fix: \href{https://github.com/UCSD-E4E/website2.0/pull/144}{merge conflict example}
    \end{itemize}
    \begin{center}
        \includegraphics[scale=.25]{merge_conflict_datacamp.png}
    \end{center}
\end{frame}

\begin{frame}{Putting it Together!}
    \begin{itemize}
        \item Lets start by adding to the blogpost!
        \item \href{https://e4e.ucsd.edu/news-and-updates/2023-summer-research-students}{Have blog post showing you were in the program!} 
        \item Send me on slack your github username
        \item Clone \href{https://github.com/UCSD-E4E/website2.0}{https://github.com/UCSD-E4E/website2.0}
        \item Checkout "2024\_reu\_students" branch
        \item make a new branch for your changes
        \item Go to \_posts/
        \item Add a photo + small blog, you can change it later
        \item photo can be dumped into "assets/people" change to notation: s.perry.jpg
        \item Commit and push those changes to github
        \item Make a PR for your changes
    \end{itemize}
\end{frame}

\begin{frame}{Extras: GitHub Actions for Automation}
    \begin{itemize}
        \item Triggered by GitHub events (push, PRs).
        \item Workflows combine actions in YAML files.
        \item Runs on GitHub-hosted or self-hosted runners.
        \item Automates tests and deployment on PR merge.
        \item Auto-assigns issues, auto-labels PRs by path.
    \end{itemize}
\end{frame}
\begin{frame}{Extras: Releases}
    \begin{itemize}
        \item Draft new release, choose git tag.
        \item Add release notes describing changes.
        \item Bundles code, executables, and assets.
        \item Detailed notes inform users of updates.
        \item Example: https://github.com/HumanSignal/label-studio/releases
    \end{itemize}
\end{frame}
\begin{frame}{Extras: Code Security}
    \begin{columns}
        \begin{column}{0.4\textwidth}
            Protection against:
            \begin{itemize}
                \item Vulnerable dependencies
                \item Some code vulnerabilities
                \item Some committed secrets
            \end{itemize}
        \end{column}
        \begin{column}{0.6\textwidth}
            \includegraphics[width=\textwidth,height=0.8\textheight,keepaspectratio]{github_security.jpg}
        \end{column}
    \end{columns}
\end{frame}
\begin{frame}{Extras: Branch Protection}
    Why do we need branch protections?
\end{frame}
\begin{frame}{Configuring Branch Protection}
    \begin{columns}
        \begin{column}{0.5\textwidth}
            \includegraphics[width=\textwidth,height=0.8\textheight,keepaspectratio]{branch_protection_1.png}
        \end{column}
        \begin{column}{0.5\textwidth}
            \includegraphics[width=\textwidth,height=0.8\textheight,keepaspectratio]{branch_protection_2.png}
        \end{column}
    \end{columns}
    \centering
\end{frame}
\begin{frame}{Branch Protection Example}
    \url{https://github.com/UCSD-E4E/branch_protections_demo}
\end{frame}



% \begin{frame}{Extras: Advanced Git}
%     \begin{itemize}
%         \item Interactive Rebase*: Rewrite history, edit commits.
%         \item Stashing*: Temporarily shelf changes for tasks.
%         \item Cherry-picking: Apply commit to another branch.
%         \item * Not recommended for shared/remote repositories.
%     \end{itemize}
% \end{frame}
% \begin{frame}{Cherry-picking}
%     \begin{itemize}
%         \item \textit{git cherry-pick commit-hash}
%     \end{itemize}
%     \begin{center}
%         \includegraphics[scale=.25]{cherry_diagram.png}
%     \end{center}
% \end{frame}


% \begin{frame}{Extra Commands}
%     \begin{itemize}
%         \item Interactive Rebase*: Rewrite history, edit commits.
%         \item Stashing: Temporarily shelf changes for tasks.
%         \item Cherry-picking: Apply commit to another branch.
%         \item * Use with caution as this action is DESTRUCTIVE.
%     \end{itemize}
% \end{frame}
% \begin{frame}{Cherry-picking}
%     \begin{itemize}
%         \item \textit{git cherry-pick commit-hash}
%     \end{itemize}
%     \begin{center}
%         \includegraphics[scale=.25]{cherry_diagram.png}
%     \end{center}
% \end{frame}
% \begin{frame}{Git Workflows}
%     \begin{itemize}
%         \item Feature Branch Workflow
%         \item Gitflow Workflow
%         \item Fork Workflow
%     \end{itemize}
% \end{frame}
% \begin{frame}{Feature Branch Workflow}
%     \begin{itemize}
%         \item Develop each feature in its own branch.
%         \item Merges via pull requests for code review.
%         \item Keeps main branch stable, encourages collaboration.
%         \item Ideal for projects with simultaneous feature development.
%     \end{itemize}
%     \begin{center}
%         \includegraphics[scale=.25]{feature_workflow_diagram.png}
%     \end{center}
% \end{frame}
% \begin{frame}{Gitflow Workflow}
%     \begin{itemize}
%         \item Structured model: development, features, releases, hotfixes.
%         \item Systematic release management, clear branch roles.
%         \item Tracks progress efficiently, supports parallel releases.
%         \item Suited for scheduled release cycles.
%     \end{itemize}
%     \begin{center}
%         \includegraphics[scale=.2]{gitflow_workflow_diagram.png}
%     \end{center}
% \end{frame}
% \begin{frame}{Fork Workflow}
%     \begin{itemize}
%         \item Developers work on personal repository copies.
%         \item Changes proposed via pull requests.
%         \item Encourages external contributions, safe experimentation.
%         \item Ideal for open-source and large collaborations.
%     \end{itemize}
%     \begin{center}
%         \includegraphics[scale=.25]{fork_workflow_diagram.png}
%     \end{center}
% \end{frame}
% \begin{frame}{Tags}
%     \begin{itemize}
%         \item Marks significant project milestones.
%         \item Useful for release points, use semantic versioning.
%         \item Lightweight tags: \textit{git tag tagname}.
%         \item Annotated tags: \textit{git tag -a tagname ``message''}.
%         \item List/delete tags: \textit{git tag}, \textit{git tag -d tagname}.
%     \end{itemize}
% \end{frame} 
% \begin{frame}{Issue Management}
%     \begin{itemize}
%         \item Templates for each type of issue.
%         \begin{itemize}
%             \item Describes issue type, steps to reproduce, etc.
%             \item Type of issue (bug, feature, etc.).
%             \item Assignees, labels.
%             \item Reproducibility, severity, and priority.
%         \end{itemize}
%     \end{itemize}
% \end{frame}   
% \begin{frame}{Pull Requests (PR) Management}
%     \begin{itemize}
%         \item Keep PRs small for easy review.
%         \item Use checklists for consistent reviews.
%         \item Automate tests and checks via GitHub Actions.
%         \item Example: https://github.com/stevemao/github-issue-templates/blob/master/checklist2/PULL\_REQUEST\_TEMPLATE.md
%     \end{itemize}
% \end{frame}
% \begin{frame}{Releases}
%     \begin{itemize}
%         \item Draft new release, choose git tag.
%         \item Add release notes describing changes.
%         \item Bundles code, executables, and assets.
%         \item Detailed notes inform users of updates.
%         \item Example: https://github.com/HumanSignal/label-studio/releases
%     \end{itemize}
% \end{frame}
% \begin{frame}{GitHub Actions for Automation}
%     \begin{itemize}
%         \item Triggered by GitHub events (push, PRs).
%         \item Workflows combine actions in YAML files.
%         \item Runs on GitHub-hosted or self-hosted runners.
%         \item Automates tests and deployment on PR merge.
%         \item Auto-assigns issues, auto-labels PRs by path.
%     \end{itemize}
% \end{frame}
% \begin{frame}{Putting it Together}
%     \begin{itemize}
%         \item Combine Tags, Releases, and GitHub Actions.
%         \item Interact with source code.
%         \item Example: https://github.com/TylerFlar/MinecraftDiscord-CrossChat
%     \end{itemize}
% \end{frame}
% \begin{frame}{Code Security}
%     \begin{columns}
%         \begin{column}{0.4\textwidth}
%             Protection against:
%             \begin{itemize}
%                 \item Vulnerable dependencies
%                 \item Some code vulnerabilities
%                 \item Some committed secrets
%             \end{itemize}
%         \end{column}
%         \begin{column}{0.6\textwidth}
%             \includegraphics[width=\textwidth,height=0.8\textheight,keepaspectratio]{github_security.jpg}
%         \end{column}
%     \end{columns}
% \end{frame}
% \begin{frame}{Branch Protection}
%     Why do we need branch protections?
% \end{frame}
% \begin{frame}{Configuring Branch Protection}
%     \begin{columns}
%         \begin{column}{0.5\textwidth}
%             \includegraphics[width=\textwidth,height=0.8\textheight,keepaspectratio]{branch_protection_1.png}
%         \end{column}
%         \begin{column}{0.5\textwidth}
%             \includegraphics[width=\textwidth,height=0.8\textheight,keepaspectratio]{branch_protection_2.png}
%         \end{column}
%     \end{columns}
%     \centering
% \end{frame}
% \begin{frame}{Branch Protection Example}
%     \url{https://github.com/UCSD-E4E/branch_protections_demo}
% \end{frame}
% \begin{frame}{More Useful GitHub Features}
%     \begin{itemize}
%         \item Projects/Project Templates
%         \item Wikis
%     \end{itemize}
% \end{frame}
\end{document}